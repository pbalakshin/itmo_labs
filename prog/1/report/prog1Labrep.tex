% Created 2019-09-09 Пн 00:07
% Intended LaTeX compiler: pdflatex
\documentclass[11pt]{article}

\usepackage[T1]{fontenc}
\usepackage{indentfirst}
\usepackage{inconsolata}
\usepackage{color}
\definecolor{pblue}{rgb}{0.13,0.13,1}
\definecolor{pgreen}{rgb}{0,0.5,0}
\definecolor{pred}{rgb}{0.9,0,0}
\definecolor{pgrey}{rgb}{0.46,0.45,0.48}
\usepackage[utf8]{inputenc}
\usepackage[T1]{fontenc}
\usepackage{graphicx}
\usepackage{grffile}
\usepackage{longtable}
\usepackage{wrapfig}
\usepackage{rotating}
\usepackage[normalem]{ulem}
\usepackage{lscape}
\usepackage{amsmath}
\usepackage{textcomp}
\usepackage{amssymb}
\usepackage{capt-of}
\usepackage{hyperref}
\usepackage[T2A]{fontenc}
\usepackage[a4paper,left=3cm,top=2cm,right=1.5cm,bottom=2cm,marginparsep=7pt,marginparwidth=.6in]{geometry}
\usepackage{cmap}
\usepackage[russian, english]{babel}
\usepackage{xcolor}
\usepackage{listings}

\usepackage{MnSymbol,wasysym}

\lstset{language=Java,
  numbers=left,
  numberstyle=\color{pgrey},
  showspaces=false,
  showtabs=false,
  breaklines=true,
  showstringspaces=false,
  breakatwhitespace=true,
  commentstyle=\color{pgreen},
  keywordstyle=\color{pblue},
  stringstyle=\color{pred},
  basicstyle=\ttfamily,
  moredelim=[il][\textcolor{pgrey}]{$$},
  moredelim=[is][\textcolor{pgrey}]{\%\%}{\%\%}
}

\author{АВТОР}
\date{\today}
\title{}
\hypersetup{
 pdfauthor={АВТОР},
 pdftitle={},
 pdfkeywords={},
 pdfsubject={},
 pdfcreator={Emacs 26.1 (Org mode 9.1.9)}, 
 pdflang={English, Russian}}
\begin{document}

\large
\thispagestyle{empty}
\begin{center}
\textbf{Санкт-Петербургский Национальный Исследовательский}\\
\textbf{Университет Информационных Технологий, Механики и Оптики}\\
\textbf{Факультет Программной Инженерии и Компьютерной Техники}\\
\end{center}
\vspace{1em}
\begin{center}
\includegraphics[width=120px]{../../../itmo-logo.png}
\end{center}
\LARGE
\vspace{5em}
\begin{center}
\textbf{Вариант № 824717}\\
\textbf{Лабораторная работa № 1}\\
\Large
\textbf{по дисциплине}\\
\LARGE
\textbf{\emph{'Программирование'}}\\
\end{center}
\vspace{11em}
\large
\begin{flushright}
\textbf{Выполнил:}\\
\textbf{Студент группы P3113}\\
\textbf{\emph{Куперштейн Дмитрий;} : 269359}\\
\textbf{Преподаватель:}\\
\textbf{\emph{Письмак Алексей Евгеньевич}}\\
\end{flushright}
\vspace{4em}
\large
\begin{center}
\textbf{Санкт-Петербург 2019 г.}
\end{center}
\pagebreak{}
\setcounter{tocdepth}{2}
\renewcommand{\contentsname}{Оглавление}
\tableofcontents
\vspace{2em}
\pagebreak{}
\section{Текст задания}
\begin{enumerate}
    \item Создать одномерный массив $a$ типа $long$. Заполнить его нечётными числами от 1 до 23 включительно в порядке убывания. 
    \item Создать одномерный массив $x$ типа $float$. Заполнить его 16-ю случайными числами в диапазоне от -2.0 до 6.0. 
    \item Создать двумерный массив $p$ размером 12x16. Вычислить его элементы по следующей формуле (где \begin{math}x = x[j]\end{math}):
        \begin{itemize}
            \item если $a[i] = 1$, то $p[i][j] = tan((e^x)^2)$
            \item если  $a[i] \in \{3, 5, 13, 15, 17, 23\}$, то $p[i][j] = (2 \cdot arcsin(\frac{1}{4} \cdot \frac{x+2}{8}))^3$
            \item для остальных значений $a[i]$: $p[i][j] = ln(sin ^2(ln((2 \cdot \frac{|x|}{2})^2)))$
        \end{itemize}
    \item Напечатать полученный в результате массив в формате с пятью знаками после запятой. 
\end{enumerate}
\pagebreak{}
\section{Исходный код программы}
\small
\lstinputlisting{../Lab1.java}
\pagebreak{}
\begin{landscape}
\section{Результат работы программы}
\footnotesize
\begin{tabular}{|c|c|c|c|c|c|c|c|c|c|c|c|c|c|c|c|c|}
\hline
&0 & 1 & 2 & 3 & 4 & 5 & 6 & 7 & 8 & 9 & 10 & 11 & 12 & 13 & 14 & 15\\
\hline
   0 & 0.00004 &   0.01585 &   0.01817 &   0.00514 &   0.00497 &   0.01018 &   0.00079 &   0.00000 &   0.02065 &   0.00705 &   0.08970 &   0.04999 &   0.07698 &   0.00336 &   0.00203 &   0.02904\\
\hline
  1 & -0.77785 &  -0.03104 &  -0.00000 &  -1.28654 &  -1.03776 &  -0.74793 &  -0.06808 &  -0.03830 &  -0.02566 &  -4.21134 &  -4.28017 &  -1.70775 &  -7.53690 &  -0.08395 &  -0.14228 &  -0.32854\\
\hline
  2 & -0.77785 &  -0.03104 &  -0.00000 &  -1.28654 &  -1.03776 &  -0.74793 &  -0.06808 &  -0.03830 &  -0.02566 &  -4.21134 &  -4.28017 &  -1.70775 &  -7.53690 &  -0.08395 &  -0.14228 &  -0.32854\\
\hline
   3 & 0.00004 &   0.01585 &   0.01817 &   0.00514 &   0.00497 &   0.01018 &   0.00079 &   0.00000 &   0.02065 &   0.00705 &   0.08970 &   0.04999 &   0.07698 &   0.00336 &   0.00203 &   0.02904\\
\hline
   4 & 0.00004 &   0.01585 &   0.01817 &   0.00514 &   0.00497 &   0.01018 &   0.00079 &   0.00000 &   0.02065 &   0.00705 &   0.08970 &   0.04999 &   0.07698 &   0.00336 &   0.00203 &   0.02904\\
\hline
   5 & 0.00004 &   0.01585 &   0.01817 &   0.00514 &   0.00497 &   0.01018 &   0.00079 &   0.00000 &   0.02065 &   0.00705 &   0.08970 &   0.04999 &   0.07698 &   0.00336 &   0.00203 &   0.02904\\
\hline
  6 & -0.77785 &  -0.03104 &  -0.00000 &  -1.28654 &  -1.03776 &  -0.74793 &  -0.06808 &  -0.03830 &  -0.02566 &  -4.21134 &  -4.28017 &  -1.70775 &  -7.53690 &  -0.08395 &  -0.14228 &  -0.32854\\
\hline
  7 & -0.77785 &  -0.03104 &  -0.00000 &  -1.28654 &  -1.03776 &  -0.74793 &  -0.06808 &  -0.03830 &  -0.02566 &  -4.21134 &  -4.28017 &  -1.70775 &  -7.53690 &  -0.08395 &  -0.14228 &  -0.32854\\
\hline
  8 & 0.77785 &  -0.03104 &  -0.00000 &  -1.28654 &  -1.03776 &  -0.74793 &  -0.06808 &  -0.03830 &  -0.02566 &  -4.21134 &  -4.28017 &  -1.70775 &  -7.53690 &  -0.08395 &  -0.14228 &  -0.32854\\
\hline
   9 & 0.00004 &   0.01585 &   0.01817 &   0.00514 &   0.00497 &   0.01018 &   0.00079 &   0.00000 &   0.02065 &   0.00705 &   0.08970 &   0.04999 &   0.07698 &   0.00336 &   0.00203 &   0.02904\\
\hline
   10 & 0.00004 &   0.01585 &   0.01817 &   0.00514 &   0.00497 &   0.01018 &   0.00079 &   0.00000 &   0.02065 &   0.00705 &   0.08970 &   0.04999 &   0.07698 &   0.00336 &   0.00203 &   0.02904\\
\hline
   11 & 0.05495 &  -1.49702 &  -2.29821 &   6.36648 &   2.17076 &  -0.26379 &   0.36968 &   0.01871 &  -0.51406 &  -1.72172 &  -1.96644 &   3.11317 & -54.52512 &  -1.36601 &   1.73777 &  -0.57262\\
\hline
\end{tabular}
\\\\\\
\large
* Результат работы программы - это форматированный под таблицу текст без ячеек, границ и номеров строк и столбцов. Таблицей приведено в отчёте для читаемости.\\\\
** Приведённая выше таблица лишь одна из множества возможных, так как результат работы зависит от псевдослучайных чисел.\\\\
\end{landscape}
\large
\section{Вывод}
В ходе этой лабораторной работы я узнал, что их надо делать, познакомился с особенностями языка Java, средствами для разработки на нём, примитивными типами данных и функциями математической библиотеки.
\par
И конечно пока верстал этот отчет немного с \text{\LaTeX}.
\par
\end{document}