
\documentclass[11pt]{article}
\usepackage[utf8]{inputenc}
\usepackage[T1]{fontenc}
\usepackage[a4paper,left=3cm,top=2cm,right=1.5cm,bottom=2cm,marginparsep=7pt,marginparwidth=.6in]{geometry}
\usepackage[russian]{babel}
\usepackage{indentfirst}
\usepackage{pscyr}
\setlength{\parindent}{5ex}
\setlength{\parskip}{1em}
\begin{document}
\section*{Отчёт об экскурсии}
\begin{flushright}
Куперштейн Дмитрий, P3113
\end{flushright}

15-го ноября вместо обычных парх пар истории науки и техники нас позвали на экскурсию в музей оптики ИТМО по адресу Биржевая линия 16. Хотелось бы сначало кратко рассказать историю здания, где располагается музей:

 В первой четверти XIX века здесь находились каменные двухэтажные склады с классицистическим фасадом. В 1824 году купец П. Е. Елисеев открыл здесь винную торговлю, и здание стало использоваться как винные склады торгового дома «Братья Елисеевы». В 1868-1869 годах архитектор Н. П. Гребёнка расширил склады, и они считались грандиозным сооружением своего времени. После того, как здания были переданы Государственному Оптическому Институту (ГОИ), в 1938--1939 годах архитекторы Б. С. Ребортович и А. И. Гегелло надстроили корпус и изменили оформление фасада, включив в его композицию скульптурные эмблемы науки и техники. В 1930-х годах здесь также находился музей потребительской кооперации (ЛСПО).

Первая часть экспозиции музея знакомит нас с историей галографии -- технологии, которая позволяет воспроизводить трёхмерные модели различных объектов. При входе в глаза сразу бросаются белые стенды, где за стеклом как будто стоят различные экспонаты из Алмазного фонда и и пасхальные яйца Карла Фаберже. Но взглянув на "экспонаты" сбоку можно удостоверится, что это обман зрения. Далее нас знакомят с различными методами получения голографических изображений.
Это экспозиция действительно производит впечатление. Было очень интересно первый раз в жизни увидеть, как чёрное полотно, при достаточном освещении воспроизводит тот или иной предмет, например бюст А.С. Пушкина!

Следующая экпозция музея знакомит нас с историей оптики, развития фотографии, истории изобретения очков и появления зеркал.

Далее нам демонстрируют, выставленные в несколько длинных рядов, блоки из различных марок стекла. Стекла подсвечиваются снизу в такт музыке. Мне очень понравилась идея этого экспоната, но было бы гораздо лучше поместить его в отдельную неосвещённую комнату с черными стенами. Тогда это действительно бы превратилось в завораживающую светомузыку!

На этом заканчивается историческая часть музея и наступает интерактивная. В начале нам рассказывают историю изучения механизма зрения и обработки информации, далее о кинематографе и 3D-видении.

После того как нам рассказывают обо всех интерактивных экспонатах наступает свободная часть экскурсии: мы можем свободно передвигаться по музею и все пощупать.

В целом у меня сложились хорошее мнение об этом музее, но впечатлили только экспонаты связанные с голографией.
\end{document}