% Created 2019-09-20 Пт 15:13
% Intended LaTeX compiler: pdflatex
\documentclass[11pt]{article}
\usepackage[utf8]{inputenc}
\usepackage[T1]{fontenc}
\usepackage{graphicx}
\usepackage{grffile}
\usepackage{longtable}

\usepackage{rotating}
\usepackage[normalem]{ulem}
\usepackage{amsmath}
\usepackage{textcomp}
\usepackage{amssymb}
\usepackage{capt-of}
\usepackage{hyperref}
\usepackage[T2A]{fontenc}
\usepackage[a4paper,left=3cm,top=2cm,right=1.5cm,bottom=2cm,marginparsep=7pt,marginparwidth=.6in]{geometry}
\usepackage{cmap}
\usepackage[russian]{babel}
\usepackage{xcolor}
\usepackage{listings}
\author{АВТОР}
\date{\today}
\title{}
\hypersetup{
 pdfauthor={АВТОР},
 pdftitle={},
 pdfkeywords={},
 pdfsubject={},
 pdfcreator={Emacs 26.1 (Org mode 9.1.9)}, 
 pdflang={Russian}}
\begin{document}
\DeclareFixedFont{\ttb}{T1}{txtt}{bx}{n}{12} % for bold
\DeclareFixedFont{\ttm}{T1}{txtt}{m}{n}{12}
\definecolor{deepblue}{rgb}{0,0,0.5}
\definecolor{deepred}{rgb}{0.6,0,0}
\definecolor{deepgreen}{rgb}{0,0.5,0}
\newcommand\pythonstyle{\lstset{
language=Python,
basicstyle=\ttm,
otherkeywords={self},             % Add keywords here
keywordstyle=\ttb\color{deepblue},
emph={MyClass,__init__},          % Custom highlighting
emphstyle=\ttb\color{deepred},    % Custom highlighting style
stringstyle=\color{deepgreen},
frame=tb,                         % Any extra options here
showstringspaces=false            % 
}}
\newcommand\pythonexternal[2][]{{
\pythonstyle
\lstinputlisting[#1]{#2}}}
\large
\thispagestyle{empty}
\begin{center}
\textbf{Санкт-Петербургский Национальный Исследовательский}\\
\textbf{Университет Информационных Технологий, Механики и Оптики}\\
\textbf{Факультет Программной Инженерии и Компьютерной Техники}\\
\end{center}
\vspace{1em}
\begin{center}
\includegraphics[width=120px]{../../../itmo-logo.png}
\end{center}
\LARGE
\vspace{5em}
\begin{center}
\textbf{Вариант № 8}\\
\textbf{Лабораторная работа № 3}\\
\Large
\textbf{по дисциплине}\\
\LARGE
\textbf{\emph{'Информатика'}}\\
\end{center}
\vspace{11em}
\large
\begin{flushright}
\textbf{Выполнил:}\\
\textbf{Студент группы P3113}\\
\textbf{\emph{Куперштейн Дмитрий;} : 269359}\\
\textbf{Преподаватель:}\\
\textbf{\emph{Малышева Татьяна Алексеевна}}\\
\end{flushright}
\vspace{4em}
\large
\begin{center}
\textbf{Санкт-Петербург 2019 г.}
\end{center}
\pagebreak{}
\setcounter{tocdepth}{2}
\tableofcontents
\vspace{2em}
\pagebreak{}
\section{Задание}
\begin{enumerate}
	\item Создать следующего вида исходный файл из десяти строк,
содержащий в каждой строке ФИО, дату рождения, дату получения
паспорта и баллы ЕГЭ по трём предметам:
	\begin{table}[htb]
	\centering
	\begin{tabular}{|l|}
	\hline
	Апельсинов А.А. 08.02.2000 17.03.2014 90 100 91\\
	Виноградов В.В. 09.03.1999 15.04.2013 67 99 98\\
	Яблоков Я.Я. 10.04.2000 19.05.2014 94 94 94\\
	Морковкин М.М. 11.05.1999 17.06.2013 91 82 73\\
	\hline
	\end{tabular}
	\end{table}
	\item Не используя готовые сторонние подключаемые функции для
факториала, int(), bin() и т.п., написать программу на языке Python
3.x, которая бы вычисляла среднее значение баллов ЕГЭ,
сортировала строки по указанной колонке в обратном порядке (от
большего к меньшему) и выводила результат следующего вида
(для сортировки по дате рождения):
	\begin{table}[htb]
	\centering
	\begin{tabular}{|l|}
	\hline
	Яблоков Я.Я. | 10.04.2000 | 19.05.2014 | 94 94 94 -> 94\\
	Апельсинов А.А. | 08.02.2000 | 17.03.2014 | 90 100 91 -> 93,666666\\
	Морковкин М.М. | 11.05.1999 | 17.06.2013 | 91 82 73 -> 82\\
	Виноградов В.В. | 09.03.1999 | 15.04.2013 | 67 99 98 -> 88\\
	\hline
	\end{tabular}
	\end{table}
	\item Написать вывод по итогам выполнения лабораторной работы.
	\item Проверить, что все пункты задания выполнены и выполнены верно.
	\item Написать отчёт о проделанной работе.
	\item Подготовиться к устным вопросам на защите\\
\end{enumerate}
По данным таблицы для варианта 8 сортировку следует выполнять по среднему значению балла ЕГЭ
\pagebreak{}
\section{Скрипт}
\footnotesize
\pythonexternal{../int_lab3.py}
\pagebreak{}
\small
\section{Содержание файла с данными}
\lstinputlisting[inputencoding=cp1251]{../input_file_1251.txt}
\section{Результат работы}
\lstinputlisting[inputencoding=cp1251]{../out_1251.txt}
\large
\section{Вывод}
В ходе этой лабораторной работы я пременил свои навыки программирования на Python3: реализовал алгоритм сортировки Хоара (быстрой сортировки, сокращённо qsort) с ключём, реализовал алгоритм перевода строки в целое
число (atoi) и обработку данных из файла по заданию.
\end{document}