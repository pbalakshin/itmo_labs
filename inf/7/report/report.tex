\documentclass[11pt,a4paper]{article}
\usepackage[utf8]{inputenc}
\usepackage{amsmath}
\usepackage{amsfonts}
\usepackage{amssymb}
\usepackage{graphicx}
\usepackage[english, russian]{babel}
\usepackage[a4paper,
            left=3cm,
            top=2cm,
            right=1.5cm,
            bottom=2cm,
            marginparsep=7pt,
            marginparwidth=.6in]{geometry}
\usepackage{indentfirst}
\begin{document}
	\begin{center}
		\section*{Текст краткого доклада о TeXstudio}
	\end{center}
	\begin{flushright}
		Подготовил студент группы P3113 Куперштейн Дмитрий, 2019
	\end{flushright}
	\subsubsection*{Слайд 1}
	«TeXstudio» — кросс-платформенный редактор \LaTeX с открытым кодом, подобный Texmaker.
	
	TeXstudio является интегрированной средой для создания \LaTeX документов и включает такие возможности, как интерактивная система проверки правописания, сворачивание блоков текста, подсветка синтаксиса. TeXstudio распространяется без пакета \LaTeX — пользователь должен самостоятельно выбрать и установить нужный дистрибутив \LaTeX.
	
	TeXstudio, который сначала назывался TexMakerX, появился как ответвление от программы Texmaker, в которой её пытались расширить с помощью дополнительной функциональности, сохраняя при этом внешний вид и поведение. Он запускается под Windows, Unix/Linux, BSD и Mac OS X.
	\subsubsection*{Слайд 2}
	TeXstudio возник как ответвление от Texmaker в 2009 году, в силу не открытого процесса разработки Texmaker и разной философии относительно конфигурирования и возможностей. Сначала он назывался TeXmakerX, потому что разрабатывался как маленький набор расширений к Texmaker с надеждой на то, что они когда-нибудь будут интегрированы в Texmaker. Несмотря на то, что в некоторых местах еще можно увидеть, что TeXstudio берет начало от Texmaker, существенные изменения в функциональности и кодовой базы делают его полностью независимой программой.
	\subsubsection*{Слайд 3}
	TeXstudio имеет много полезных возможностей, необходимых при редактировании исходного кода \TeX/\LaTeX, таких как:
	\begin{itemize}
		\item Поддержка написания скриптов
		\item Мастер для рисунков, таблиц, формул (мастера используют GUI для визуального и \item удобного для пользователя создания структур со сложным кодом)
		\item Поддержка Drag \& Drop для рисунков
		\item Система шаблонов
		\item Подсветка синтаксиса
		\item Проверка правописания
		\item Интегрированный просмотрщик PDF
		\item Автоматически обновляемый просмотр для формул и сегментов кода в месте редактирования
		\item Поддержка SVN
		\item Интеграция с библиографическими менеджерами BibTeX и BibLaTeX
		\item Экспорт в формат HTML
		\item Лексический анализ документа (количество слов, частота слов, частота фраз и тому подобное)
	\end{itemize}
	\newpage
	\subsubsection*{Практика}
	В интерфейсе редактора TeXStudio с помощью мастера создать шаблон для простого \LaTeX файла. Для этого перейти в меню Wizards -> Quick Start, использовть стандартные настройки. После создания шаблона подключить пакет babel для русского и английского языка командой \texttt{\textbackslash usepackage[english,russian]\{babel\}}.
	
	Продемонстрировать проверку орфографии, стандартные сочетания клавиш для смены начертания и работы в математическом режиме, меню быстрого доступа к некоторым командам, инструменты для работы с таблицами. 
	
	Попробовать собрать в TeXStudio файл, использующий функционал ХеLaTeX, изучить сообщение об ошибке. Прописать в начале документа магический комментарий (Magic comment) \texttt{\% !TeX program = xelatex}. Продемонстрировать успешную сборку документа.
	
	\subsubsection*{Заключение}
	Редактор TeXstudio по-настоящему полезен при создании документов в \TeX. Его мощь действительно раскрывается при работе с документами сложной структуры.
\end{document}