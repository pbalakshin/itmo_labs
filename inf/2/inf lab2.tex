% Created 2019-09-20 Пт 15:13
% Intended LaTeX compiler: pdflatex
\documentclass[11pt]{article}
\usepackage[utf8]{inputenc}
\usepackage[T1]{fontenc}
\usepackage{graphicx}
\usepackage{grffile}
\usepackage{longtable}
\usepackage{wrapfig}
\usepackage{rotating}
\usepackage[normalem]{ulem}
\usepackage{float}
\usepackage{multirow,multicol, makecell, booktabs}
\usepackage{amsmath}
\usepackage{textcomp}
\usepackage{amssymb}
\usepackage{capt-of}
\usepackage{hyperref}
\usepackage[T2A]{fontenc}
\usepackage[a4paper,left=3cm,top=2cm,right=1.5cm,bottom=2cm,marginparsep=7pt,marginparwidth=.6in]{geometry}
\usepackage{cmap}
\usepackage[russian]{babel}
\usepackage{xcolor}
\usepackage{listings}
\usepackage{paracol}

\author{АВТОР}
\date{\today}
\title{}
\hypersetup{
 pdfauthor={АВТОР},
 pdftitle={},
 pdfkeywords={},
 pdfsubject={},
 pdfcreator={Emacs 26.1 (Org mode 9.1.9)}, 
 pdflang={Russian}}
\begin{document}

\large
\thispagestyle{empty}
\begin{center}
\textbf{Санкт-Петербургский Национальный Исследовательский}\\
\textbf{Университет Информационных Технологий, Механики и Оптики}\\
\textbf{Факультет Программной Инженерии и Компьютерной Техники}\\
\end{center}
\vspace{1em}
\begin{center}
\includegraphics[width=120px]{../../itmo-logo.png}
\end{center}
\LARGE
\vspace{5em}
\begin{center}
\textbf{Вариант № 20}\\
\textbf{Лабораторная работа № 2}\\
\Large
\textbf{по дисциплине}\\
\LARGE
\textbf{\emph{'Информатика'}}\\
\end{center}
\vspace{11em}
\large
\begin{flushright}
\textbf{Выполнил:}\\
\textbf{Студент группы P3113}\\
\textbf{\emph{Куперштейн Дмитрий;} : 269359}\\
\textbf{Преподаватель:}\\
\textbf{\emph{МАЛЫШЕВА ТАТЬЯНА АЛЕКСЕЕВНА}}\\
\end{flushright}
\vspace{4em}
\large
\begin{center}
\textbf{Санкт-Петербург 2019 г.}
\end{center}
\pagebreak{}
\setcounter{tocdepth}{2}
\tableofcontents
\vspace{2em}
\pagebreak{}
\section{Задание}
\begin{enumerate}
	\item Переписать в отчёт (рукой, а не копированием в электронном виде)
формулировку заданий 4–10! Это требуется для того, чтобы
корректно и в полном объёме выполнить все необходимые пункты
задания. Данную лабораторную надо выполнять как вычислительная
машина, которая действует строго по инструкции.
	\item Определить свои числа А и С исходя из варианта. Вариант выбирается
как сумма последнего числа в номере группы и номера в списке
группы согласно ISU.
	\item По заданному варианту исходных данных получить набор десятичных
чисел:\\
X1 = A, X2 = C,\\
X3 = A+C, X4 = A+C+C, X5 = C-A, X6 = 65536-X4,\\
X7 = -X1, X8 = -X2, X9 = -X3, X10 = -X4, X11 = -X5, X12 = -X6.
	\item Выполнить перевод десятичных чисел X1,…,X6 в двоичную систему
счисления, получив их двоичные эквиваленты B1,…,B6
соответственно.
Не использовать при этом никакой формат представления данных, не
использовать никакую разрядную сетку.
	\item Используя 16-разрядный двоичный формат со знаком и полученные в
предыдущем пункте задания двоичные числа B1,…,B6 (т.е. при
необходимости дополнить числа B1…B6 ведущими нулями и
однозначно интерпретировать эти числа в 16-разрядном двоичном
формате со знаком), вычислить двоичные числа B7,…,B12: B7 = -B1,
B8 = -B2, B9 = -B3, B10 = -B4, B11 = -B5, B12 = -B6. Отрицательные
числа представлять в дополнительном коде.
	\item Найти область допустимых значений для данного двоичного формата.
	\item Выполнить обратный перевод всех двоичных чисел B1…B12
(используя 16-разрядный двоичный формат со знаком) в десятичные
и прокомментировать полученные результаты.
	\item Выполнить следующие сложения двоичных чисел:
B1+B2, B2+B3, B2+B7, B7+B8, B8+B9, B1+B8, B11+B3 (итого, 7
операций сложения).
Для представления слагаемых и результатов сложения использовать
16-разрядный двоичный формат со знаком. Результаты сложения
перевести в десятичную систему счисления, сравнить с
соответствующими десятичными числами (т.е. сравнить с суммой
слагаемых, представленных в десятичной системе: B1 + B2 vs X1 +
X2).
	\item В отчёте (письменно, а не устно при ответе) дать подробные
комментарии полученным результатам (к каждому результату
сложения), как показано в таблице 2.6 книги «Введение в
микроЭВМ». Расставить 6 флагов состояния.
	\item При выставлении вспомогательного флага переноса (межтетрадный
перенос – AF = Auxiliary Carry Flag) учитывать перенос не между 7-м и
8-м битами, а между 3-м и 4-м битами результата. При выставлении
флага чётности PF учитывать только младший байт.
	\item Проверить, что все пункты задания выполнены и выполнены верно.
\end{enumerate}
\pagebreak{}
\section{Решение}
\begin{paracol}{2}
\begin{enumerate}
\setcounter{enumi}{2}
	\item
		A = 5567\\
		C = 26281\\
		X1 =  A = 5567\\
		X2 = C = 26281\\
		X3 = A + C =31848\\
		X4 = A + C + C = 58129\\
		X5 = C - A = 20714\\
		X6 = 65536-X4 = 7407\\
		X7 = -X1 = $-$5567\\
		X8 = -X2 = $-$26281\\
		X9 = -X3 = $-$31848\\
		X10 = -X4 = $-$58129\\
		X11 = -X5 = $-$20714\\
		X12 = -X6 = $-$7407
	\item 
		X1$_{(10)}$ $\rightarrow$ B1$_{(2)}$ = \phantom{000}1 0101 1011 1111\\
		X2$_{(10)}$ $\rightarrow$ B2$_{(2)}$ = \phantom{0}110 0110 1010 1001\\
		X3$_{(10)}$ $\rightarrow$ B3$_{(2)}$ = \phantom{0}111 1100 0110 1000\\
		X4$_{(10)}$ $\rightarrow$ B4$_{(2)}$ = 1110 0011 0001 0001\\
		X5$_{(10)}$ $\rightarrow$ B5$_{(2)}$ =  \phantom{0}101 0000 1110 1010\\
		X6$_{(10)}$ $\rightarrow$ B6$_{(2)}$ = \phantom{000}1 1100 1110 1111
	\item
		B1$_{(2)}$ \phantom{0= $-$B1$_{(2)}$ }= 0001 0101 1011 1111\\
		B2$_{(2)}$ \phantom{0= $-$B1$_{(2)}$ }= 0110 0110 1010 1001\\
		B3$_{(2)}$ \phantom{0= $-$B1$_{(2)}$ }= 0111 1100 0110 1000\\
		B4$_{(2)}$ \phantom{0= $-$B1$_{(2)}$ }= 1110 0011 0001 0001\\
		B5$_{(2)}$ \phantom{0= $-$B1$_{(2)}$ }= 0101 0000 1110 1010\\
		B6$_{(2)}$ \phantom{0= $-$B1$_{(2)}$ }= 0001 1100 1110 1111\\
		B7$_{(2)}$ \phantom{0}= $-$B1$_{(2)}$ = 1110 1010 0100 0001\\
		B8$_{(2)}$\phantom{0} = $-$B2$_{(2)}$ = 1001 1001 0101 0111\\
		B9$_{(2)}$ \phantom{0}= $-$B3$_{(2)}$ = 1000 0011 1001 1000\\
		B10$_{(2)}$ = $-$B4$_{(2)}$ = 0001 1100 1110 1111\\
		B11$_{(2)}$ = $-$B5$_{(2)}$ = 1010 1111 0001 0110\\
		B12$_{(2)}$ = $-$B6$_{(2)}$ = 1110 0011 0001 0001
	\item $[-32768,32767]$
	\item 
		B1$_{(2)}$\phantom{0} $\rightarrow$ Y1$_{(10)}$\phantom{0} = X1$_{(10)}$\\
		B2$_{(2)}$\phantom{0} $\rightarrow$ Y2$_{(10)}$\phantom{0} = X2$_{(10)}$\\
		B3$_{(2)}$\phantom{0} $\rightarrow$ Y3$_{(10)}$\phantom{0} = X3$_{(10)}$\\
		B4$_{(2)}$\phantom{0} $\rightarrow$ Y4$_{(10)}$\phantom{0} $\neq$ X4$_{(10)}$\\
		B5$_{(2)}$\phantom{0} $\rightarrow$ Y5$_{(10)}$\phantom{0} = X5$_{(10)}$\\
		B6$_{(2)}$\phantom{0} $\rightarrow$ Y6$_{(10)}$\phantom{0} = X6$_{(10)}$\\
		B7$_{(2)}$\phantom{0} $\rightarrow$ Y7$_{(10)}$\phantom{0} = X7$_{(10)}$\\
		B8$_{(2)}$\phantom{0} $\rightarrow$ Y8$_{(10)}$\phantom{0} = X8$_{(10)}$\\
		B9$_{(2)}$\phantom{0} $\rightarrow$ Y9$_{(10)}$\phantom{0} = X9$_{(10)}$\\
		B10$_{(2)}$ $\rightarrow$ Y10$_{(10)}$ $\neq$ X10$_{(10)}$\\
		B11$_{(2)}$ $\rightarrow$ Y11$_{(10)}$ = X11$_{(10)}$\\
		B12$_{(2)}$ $\rightarrow$ Y12$_{(10)}$ = X12$_{(10)}$
	\switchcolumn
	\setcounter{enumi}{7}
		Результат обратного перевода из двоичного числа в десятичное равен исходному десятичному числу
		в 11 случаях из 12 (B4 $\neq$ X4).\\
	\item 
		\begin{tabular}[t]{cccccc}
			&       &                                                & 1111   & \phantom{1}111 &111\phantom{1}\\
			\multirow{2}{*}{$+$} & B1$_{(2)}$ & 0001 & 0101 & 1011 & 1111\\
						          & B2$_{(2)}$ & 0110 & 0110 & 1010 &  1001\\
			\cline{2-6}
						       &        & 0111 & 1100 & 0110 & 1000\\
		\\
		\end{tabular}\\
		= 31848$_{(10)}$ =  X3$_{(10)}$ = X1$_{(10)}$ + X2$_{(10)}$\\
\\
		\begin{tabular}{ccc}
			SF = 0 & ZF = 0 & PF = 0 \\
			AF = 1 & CF = 0 & OF = 0\\
		\end{tabular}\\\\
		Результат корректный.\\
	\\\\
		\begin{tabular}[t]{cccccc}
			&&1111&1\phantom{11}1&11\phantom{1}1&\\
			\multirow{2}{*}{$+$} & B2$_{(2)}$ & 0110 & 0110 & 1010 & 1001\\
						            & B3$_{(2)}$ & 0111 & 1100 & 0110 &  1000\\
			\cline{2-6}
						       &        & 1110 & 0011 & 0001 & 0001\\
		\\
		\end{tabular}\\
		= $-$7407$_{(10)} \neq$ X2$_{(10)}$ + X3$_{(10)}$\\ 
		\phantom{=} $-7407_{(10)}$ $\neq$ X4$_{(10)}$(58129)\\\\ 
		\begin{tabular}{ccc}
			SF = 1 & ZF = 0 & PF = 1 \\
			AF = 1 & CF = 0 & OF = 1\\
		\end{tabular}\\\\
		При сложении положительных чисел полуен отрицательный результат -- ПЕРЕПОЛНЕНИЕ!\\\\
		Когда происходит переполнение при сложении положительных чисел в формате n разрядов со знаком, результат можно вычислить в в десятичной системе счисления,
		прибавив к $-2^n$ ожидаемый результат:\\\\
		$-2^{16} + 58129 = -7407 =$ X12$_{(10)}$\\\\
		Это объясняется тем, что так как знаковый бит результата равен еденице (отрицательное число), то его дополнительный код равен (где $x$ прямой код):\\\\
		$(2^n-1) -(x \bmod 2^n) + 1$\\\\\\
		\switchcolumn
		\pagebreak{}
		\begin{tabular}[t]{cccccc}
			&&11\phantom{1}1&11\phantom{11}&&\phantom{11}1\phantom{1}\\
			\multirow{2}{*}{$+$} & B2$_{(2)}$ & 0110 & 0110 & 1010 & 1001\\
						            & B7$_{(2)}$ & 1110 & 1010 & 0100 & 0001\\
			\cline{2-6}
						        &   \phantom{000}1     & 0101 & 0000 & 1110 & 1010\\
		\\
		\end{tabular}\\
		= $20714_{(10)} = $ X5$_{(10)}$ = X2$_{(10)}$ + X7$_{(10)}$\\\\
		\begin{tabular}{ccc}
			SF = 0 & ZF = 0 & PF = 0 \\
			AF = 0 & CF = 1 & OF = 0\\
		\end{tabular}\\\\
		Результат корректный. Перенос из старшего разряда не учитывается.
		\\\\\\
		\begin{tabular}[t]{cccccc}
			&&1111&&1\phantom{111}&111\phantom{1}\\
			\multirow{2}{*}{$+$}  & B7$_{(2)}$ & 1110 & 1010 & 0100 & 0001\\
						            & B8$_{(2)}$ & 1001 & 1001 & 0101 & 0111\\
			\cline{2-6}
						       &   1     & 1000 & 0011 & 1001&  1000\\
		\\
		\end{tabular}\\
		= $-31848_{(10)}$ = X9$_{(10)}$ = X7$_{(10)}$ + X8$_{(10)}$\\\\
		\begin{tabular}{ccc}
			SF = 1 & ZF = 0 & PF = 0 \\
			AF = 0 & CF = 1 & OF = 0\\
		\end{tabular}\\\\
		Результат корректный. Перенос из старшего разряда не учитывается.
		\\\\\\
		\begin{tabular}[t]{cccccc}
			&&&\phantom{1}11\phantom{1}&\phantom{11}1\phantom{1}&\\
			\multirow{2}{*}{$+$}  & B8$_{(2)}$ & 1001 & 1001 & 0101 & 0111\\
						            & B9$_{(2)}$ & 1000 & 0011 &  1001 & 1000\\
			\cline{2-6}
						       &   1     & 0001 & 1100 & 1110 & 1111\\
		\\
		\end{tabular}\\
		= $7407_{(10)} \neq$ X8$_{(10)}$ + X9$_{(10)}$\\
		\phantom{=} $7407_{(10)}$ $\neq$ X10$_{(10)}$ ($-$58129)\\\\
		\begin{tabular}{ccc}
			SF = 0 & ZF = 0 & PF = 0 \\
			AF = 0 & CF = 1 & OF = 1\\
		\end{tabular}\\\\
		При сложении отрицательных чисел полуен положительный результат -- ПЕРЕПОЛНЕНИЕ!\\\\
		Когда происходит переполнение при сложении отрицательных чисел в формате n разрядов со знаком, результат можно вычислить в в десятичной системе счисления,
		прибавив к $2^n$ ожидаемый результат:\\\\
		\switchcolumn
		$2^{16} - 58129 = 7407 =$ X16$_{(10)}$\\\\
		Это объясняется тем, что так как знаковый бит результата равен нулю (положтельное число), то его дополнительный код равен прямому.\\\\\\
		\begin{tabular}[t]{cccccc}
			&       &                                                \phantom{11}1\phantom{1}&\phantom{11}11   &1111 &111\phantom{1}\\
			\multirow{2}{*}{$+$} & B1$_{(2)}$ & 0001 & 0101 & 1011 & 1111\\
						          & B8$_{(2)}$ & 1001 & 1001 & 0101 & 0111\\
			\cline{2-6}
						       &        & 1010 & 1111 & 0001 & 0110\\
		\\
		\end{tabular}\\
		=$ -20714_{(10)}$\\
		\phantom{=}$ -20714_{(10)}$  = X1$_{(10)}$ $+$ X9$_{(10)}$ ($-20714$)\\\\
		SF = 1\qquad ZF = 0\qquad PF = 0 \\
		AF = 1\qquad CF = 1\qquad OF = 0\\\\\\
		\begin{tabular}[t]{cccccc}
			&       &                                                1111&1\phantom{111}   &&\\
			\multirow{2}{*}{$+$} & B11$_{(2)}$ & 1010 & 1111 & 0001 & 0110\\
						          & B3$_{(2)}$\phantom{1} &  0111 &  1100 & 0110 & 1000\\
			\cline{2-6}
						       &        1& 0010 & 1011 & 0111 & 1110\\
		\\
		\end{tabular}\\
		=$11134_{(10)}$\\
		\phantom{=}$ 11134_{(10)}$  = X11$_{(10)}$ $+$ X3$_{(10)}$ ($11134$)\\\\
		SF = 0\qquad ZF = 0\qquad PF = 1\\
		AF = 0\qquad CF = 1\qquad OF = 0\\\\\\
\end{enumerate}
\end{paracol}
\section{Вывод}
В ходе этой лабораторной работы я понял, что быть болванчиком достаточно заморочно. Вместого того, чтобы дать задания студентам
обосновать эти алгоритмы и само наличие доп кодов вы им говорите просто посчитать. Найс потратил время, если коротко.
\end{document}