% Created 2019-09-20 Пт 15:13
% Intended LaTeX compiler: pdflatex
\documentclass[11pt]{article}
\usepackage{makecell}
\usepackage{multirow}

\usepackage[utf8]{inputenc}
\usepackage[T1]{fontenc}
\usepackage{graphicx}
\usepackage{grffile}
\usepackage{longtable}
\usepackage{wrapfig}
\usepackage{rotating}
\usepackage[normalem]{ulem}
\usepackage{amsmath}
\usepackage{textcomp}
\usepackage{amssymb}
\usepackage{capt-of}
\usepackage{hyperref}
\usepackage[T2A]{fontenc}
\usepackage[a4paper,left=3cm,top=2cm,right=1.5cm,bottom=2cm,marginparsep=7pt,marginparwidth=.6in]{geometry}
\usepackage{cmap}
\usepackage[russian]{babel}
\usepackage{xcolor}
\usepackage{listings}
\usepackage{makecell}
\newcommand{\innp}[1]{\langle #1\rangle}
\author{АВТОР}
\date{\today}
\title{}
\hypersetup{
 pdfauthor={АВТОР},
 pdftitle={},
 pdfkeywords={},
 pdfsubject={},
 pdfcreator={Emacs 26.1 (Org mode 9.1.9)}, 
 pdflang={Russian}}
\begin{document}

\scriptsize
\thispagestyle{empty}
\begin{center}
\begin{tabular}{ c c c }
\raisebox{7ex}{
\makecell{
\scriptsize
\textbf{Санкт-Петербургский Национальный Исследовательский Университет} \\
 	  \textbf{Информационных Технологий, Механики и Оптики}\\
\small
\\ \textbf{КАФЕДРА ФИЗИКИ}}} &
\raisebox{3ex}{
\includegraphics[height=10ex]{../../../itmo-logo.png}
} & \raisebox{7ex}{\makecell{\textbf{УНИВЕРСИТЕТ} \\ \textbf{ИТМО}}} \\[-2ex]
\end{tabular}
\noindent\rule{\textwidth}{1.5pt}\\
\Large
\begin{tabular}{ p{7.8cm} p{7.8cm} }
\\
Группа\hrulefill P3113 \hrulefill & К работе допущен\hrulefill\\[+0.3cm]
Студенты \makecell{Куперштейн Дмитрий и\\ Александр Машковцев}  & Работа выполнена\hrulefill\\[+0.3cm]
Преподаватель \hrulefill Боярский К.К \hrulefill & Отчет принят\hrulefill\\[+1cm]
\end{tabular}
\Huge
\textbf{Рабочий протокол и отчет по} \\
\textbf{лабораторной работе №}\\
\hrulefillИсследование распределения\hrulefill\\
\hrulefill случайной величины\hrulefill \\
\end{center}
\pagebreak{}
\large
\tableofcontents{}
\pagebreak{}
\section{Цель работы}
\begin{enumerate}
	\item Провести многократные измерения определенного интервала времени.
	\item Построить гистограмму распределения результатов измерения.
	\item Вычислить среднее значение и дисперсию полученной выборки.
	\item Сравнить гистограмму с графиком функции Гаусса с такими же
как и у экспериментального распределения средним значением и
дисперсией.
\end{enumerate}
\section{Введение}
Один из студентов
запускал механический секундомер, отдавая команду 'Старт!'  и ожидал пока стрелка достигнет отметки в пять секунд, отмечая это командой 'Стоп!'. Второй студент
руководствуясь командами первого замерял промежуток времени между командами 'Старт!' и 'Стоп!', используя цифровой секундомер, встроенный в мобильный телефон.
В результате этих измерений был получен массив 50-ти чисел (не одно из которых не совпадает ровно с пятью секундами). Для этих 50-ти измерений построены графики и проведены расчеты.\\
\section{Ход работы}
\large
\begin{enumerate}
	\item ~
		\begin{table}[htb]
		\large
		\centering
		\caption{Результаты прямых измерений}
		\begin{tabular}{|c|c|c|c|}
			\hline
			№ & $t_i, c$ & $t_i - \innp{t}_N, c$ & $(t_i - \innp{t}_N)^2$\\
			\hline
			1 &4.97 & 0.21 & 0.04\\
			\hline
			2 &4.81 & 0.05 & 0.00\\
			\hline
			3 &4.65 & -0.11 & 0.01\\
			\hline
			4 &4.51 & -0.25 & 0.06\\
			\hline
			5 &4.76 & 0.00 & 0.00\\
			\hline
			6 &4.81 & 0.05 & 0.00\\
			\hline
			7 &4.57 & -0.19 & 0.04\\
			\hline
			8 &4.80 & 0.04 & 0.00\\
			\hline
			9 &4.54 & -0.22 & 0.05\\
			\hline
			10 &4.78 & 0.02 & 0.00\\
			\hline
			11 &4.81 & 0.05 & 0.00\\
			\hline
			12 &5.02 & 0.26 & 0.07\\
			\hline
			13 &4.84 & 0.08 & 0.01\\
			\hline
			14 &4.87 & 0.11 & 0.01\\
			\hline
			15 &4.82 & 0.06 & 0.00\\
			\hline
			16 &4.60 & -0.16 & 0.03\\
			\hline
			17 &4.67 & -0.09 & 0.01\\
			\hline
			18 &4.81 & 0.05 & 0.00\\
			\hline
			19 &4.91 & 0.15 & 0.02\\
			\hline
			20 &4.56 & -0.20 & 0.04\\
			\hline
			21 &4.52 & -0.24 & 0.06\\
			\hline
			22 &4.80 & 0.04 & 0.00\\
			\hline
			23 &4.72 & -0.04 & 0.00\\
			\hline
		\end{tabular}
		\end{table}
\pagebreak{}
\begin{table}[htb]
	\large
	\centering
	\begin{tabular}{|c|c|c|c|}
	\hline
	№ & $t_i, c$ & $t_i - \innp{t}_N, c$ & $(t_i - \innp{t}_N)^2$\\
	\hline
	24 &4.64 & -0.12 & 0.01\\
	\hline
	25 &4.71 & -0.05 & 0.00\\
	\hline
	26 &4.45 & -0.31 & 0.10\\
	\hline
	27 &4.77 & 0.01 & 0.00\\
	\hline
	28 &4.87 & 0.11 & 0.01\\
	\hline
	29 &4.83 & 0.07 & 0.00\\
	\hline
	30 &4.67 & -0.09 & 0.01\\
	\hline
	31 &4.70 & -0.06 & 0.00\\
	\hline
	32 &4.71 & -0.05 & 0.00\\
	\hline
	33 &4.99 & 0.23 & 0.05\\
	\hline
	34 &5.12 & 0.36 & 0.13\\
	\hline
	35 &4.72 & -0.04 & 0.00\\
	\hline
	36 &4.81 & 0.05 & 0.00\\
	\hline
	37 &4.52 & -0.24 & 0.06\\
	\hline
	38 &4.82 & 0.06 & 0.00\\
	\hline
	39 &4.81 & 0.05 & 0.00\\
	\hline
	40 &4.70 & -0.06 & 0.00\\
	\hline
	41 &4.38 & -0.38 & 0.14\\
	\hline
	42 &4.92 & 0.16 & 0.03\\
	\hline
	43 &4.74 & -0.02 & 0.00\\
	\hline
	44 &4.67 & -0.09 & 0.01\\
	\hline
	45 &4.67 & -0.09 & 0.01\\
	\hline
	46 &4.96 & 0.20 & 0.04\\
	\hline
	47 &4.78 & 0.02 & 0.00\\
	\hline
	48 &4.89 & 0.13 & 0.02\\
	\hline
	49 &5.10 & 0.34 & 0.12\\
	\hline
	50 &4.71 & -0.05 & 0.00\\
	\hline
	& $\innp{t} = 4.76, c$ & $\sum\limits_{i=1}^N(t_i - \innp{t}_N) = -0.19, c$ & \makecell{$\sigma_N = 0.15$ \\ $\rho_{max} = 2.66$}\\
	\hline
	\end{tabular}
\end{table}
\item Для полученных измерений была построенна гистрограмма с графиком распределения Гаусса (приложение 1). Все параметры гистограммы приведены  ниже.
Максимальное и минимальное замеренное время: $$t_{min} = 4.38$$ $$t_{max} = 5.12$$
Отрезок $[t_{min}; t_{max}]$ разбит на $m = 10$ равных интервалов $\Delta t = 0.074$
\pagebreak{}
\begin{table}[htb]
	\caption{Данные для построения гистрограммы}
	\centering
	\Large
	\begin{tabular}{|c|c|c|c|c|}
	\hline
	\makecell{Границы\\интервалов, с} & $\Delta N$ & $\frac{\Delta N}{N\Delta t}, c^{-1}$ & t, c& $\rho, c^{-1}$\\
	\hline
	4.380 & 2 & 0.54 & 4.42 & 0.19\\
	\cline{1-1}
	4.454 & & & &\\
	\hline
	4.454 & 3 & 0.81 & 4.49 & 0.53\\
	\cline{1-1}
	4.528 & & & &\\
	\hline
	4.528 & 4 & 1.08 & 4.56 & 1.14\\
	\cline{1-1}
	4.602 & & & &\\
	\hline
	4.602 & 6 & 1.62 & 4.64 & 1.92\\
	\cline{1-1}
	4.676 & & & &\\
	\hline
	4.676 & 8 & 2.16 & 4.71 & 2.53\\
	\cline{1-1}
	4.750 & & & &\\
	\hline
	4.750 & 14 & 3.78 & 4.79 & 2.62\\
	\cline{1-1}
	4.824 & & & &\\
	\hline
	4.824 & 5 & 1.35 & 4.86 & 2.12\\
	\cline{1-1}
	4.898 & & & &\\
	\hline
	4.898 & 4 & 1.08 & 4.93 & 1.35\\
	\cline{1-1}
	4.972 & & & &\\
	\hline
	4.972 & 2 & 0.54 & 5.01 & 0.67\\
	\cline{1-1}
	5.046 & & & &\\
	\hline
	5.046 & 2 & 0.54 & 5.08 & 0.26\\
	\cline{1-1}
	5.120 & & & &\\
	\hline
	\end{tabular}
	\caption{Данные для построения гистрограммы}
	\begin{tabular}{|c|c|c|c|c|c|}
	\hline
	& \multicolumn{2}{c|}{Интервал, с} & \multirow{2}{*}{$\Delta N$} & \multirow{2}{*}{$\frac{\Delta N}{N}$} &  \multirow{2}{*}{P}\\
	\cline{2-3}
	&  от & до & & &\\
	\hline
	$\innp{t}_N \pm \sigma_N$ & 4.61 & 4.91 & 34& 0.68& 0.683\\
	\hline
	$\innp{t}_N \pm 2\sigma_N$ & 4.46 & 5.06 & 46& 0.92& 0.954\\
	\hline
	$\innp{t}_N \pm 3\sigma_N$ & 4.11 & 5.21 & 50& 1.00& 0.997\\
	\hline
	\end{tabular}
\end{table}
\item
	Находим табличное значения коэффициента Стьюдента  $t_{\alpha, N}$ для
доверительной вероятности $\alpha = 0.95$: $ t_{\alpha, N} = 2.0086$.
Доверительный интервал для измеряемого в работе промежутка времени:
$$ \Delta t =  t_{\alpha, N} \cdot \sigma_{\innp{t}} = 0.30$$
\end{enumerate}
\end{document}