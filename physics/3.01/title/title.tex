
\documentclass[11pt]{article}
\usepackage[utf8]{inputenc}
\usepackage[T1]{fontenc}
\usepackage[a4paper,left=3cm,top=2cm,right=1.5cm,bottom=2cm,marginparsep=7pt,marginparwidth=.6in]{geometry}
\usepackage[russian]{babel}
\usepackage{indentfirst}
\usepackage{pscyr}
\usepackage{makecell}
\usepackage{graphicx}
\setlength{\parindent}{5ex}
\setlength{\parskip}{1em}
\begin{document}
\scriptsize
\thispagestyle{empty}
\begin{center}
	\begin{tabular}{ c c c }
		\raisebox{7ex}{
			\makecell{
				\small
				\textbf{Национальный Исследовательский Университет ИТМО}\\
				\\ \textbf{КАФЕДРА ФИЗИКИ}}} &
		\raisebox{3ex}{
			\includegraphics[height=10ex]{../../../itmo-logo.png}
		} & \raisebox{7ex}{\makecell{\textbf{УНИВЕРСИТЕТ} \\ \textbf{ИТМО}}} \\[-2ex]
	\end{tabular}
	\noindent\rule{\textwidth}{1.5pt}\\
	\Large
	\begin{tabular}{ p{7.8cm} p{7.8cm} }
		\\
		Группа\hrulefill P3113\hrulefill& К работе допущен\hrulefill\\[+0.3cm]
		Студент\hrulefill Куперштейн Д.П.\hrulefill& Работа выполнена\hrulefill\\[+0.3cm]
		Преподаватель\hrulefill Боярский К.К.\hrulefill& Отчет принят\hrulefill\\[+1cm]
	\end{tabular}
	\Huge
	\textbf{Рабочий протокол и отчет по} \\
	\textbf{лабораторной работе № 3.01}\\
	\hrulefill Изучение электростатического поля\hrulefill\\
	\hrulefill методом моделирования\hrulefill\\
\end{center}
\pagebreak{}
\end{document}